
\documentclass[authorinfoot]{fidius-p}
\usepackage{listings}
\usepackage{fancyvrb}
% \usepackage{tabularx}
\author{BK \and DM}
% \title[short title]{long title}
% optional: \subtitle[short subtitle]{long subtitle}
\title{C\&C - Server}
\subtitle{Zwischenpräsentation}
\date{10.12.2010}

\setcounter{tocdepth}{1}

\setbeamertemplate{bibliography item}[text]

\setbeamertemplate{bibliography entry title}{}
\setbeamertemplate{bibliography entry location}{}
\setbeamertemplate{bibliography entry note}{}

\bibliographystyle{is-plain}

\begin{document}

\frame{\titlepage}
\frame{\tableofcontents}

\section{Ergebnisse}

\begin{frame}[fragile,containsverbatim]
  \frametitle{Ergebnisse}
  \begin{lstlisting}
    (Date.new(2010,12,10)-Date.new(2010,11,12)).to_i
     => 28
  \end{lstlisting}
  \begin{itemize}
   \item Oberfläche verschönert (CSS)
   \item Integration des Msf-Workers in GUI
   \item Anbindung an Prelude (ActiveRecord)
   \item Msf-Plugin, um Payload von Exploits zu loggen
   \item TCPDump, um Payload von NMap zu loggen
   \item Ergebnisdarstellung in GUI
  \end{itemize}

\end{frame}


\section{Ausblick}
\secframe{Ausblick}{
  \begin{itemize}
   \item Integration von CVE
   \item Hosts und Pivots visualisieren
   \item Auf Florians Testlandschaft laufen lassen
   \item Gewonnene Information verwenden, um z.B. "'leiser"' anzugreifen
  \end{itemize}
}


\section{Retrospektive}
\secframe{Retrospektive}{
  \begin{itemize}
   \item Einstieg in Git nicht ganz leicht
   \item Redmine wieder geweckt
   \item Eindruck: kleinere Teams dynamischer und schaffen mehr !?
  \end{itemize}

}

\end{document}

