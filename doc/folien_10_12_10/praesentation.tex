\documentclass[authorinfoot]{fidius-p}
\usepackage{listings}
\lstset{
  language=Ruby,
  upquote=true,
  tabsize=4,
  basicstyle=\ttfamily\fontsize{8}{14}\selectfont,
  keywordstyle=\bfseries\color{Strong Blue},
  morekeywords={Date},
  commentstyle=\itshape\color{Grey 4},
  stringstyle=\color{Red},
  backgroundcolor=\color{Beige!50},
  columns=fixed,
  showstringspaces=false,
  framexleftmargin=1em,
  framexrightmargin=1em
}
\author{BK \and DM}
\title{C\&C-Server}
\subtitle{Zwischenpräsentation}
\date{10.12.2010}

\setcounter{tocdepth}{1}

\begin{document}

\frame{\titlepage}
\frame{\tableofcontents}

\section{Ergebnisse}
\subsection{Ergebnisse}
\begin{frame}[fragile,containsverbatim]
  \frametitle{Ergebnisse}
  \begin{lstlisting}[gobble=4]
    (Date.new(2010,12,10) - Date.new(2010,11,12)).to_i   #=> 28
  \end{lstlisting}
  \begin{itemize}
    \item Oberfläche verschönert (CSS)
    \item Integration des MSF-Workers in GUI
    \item Anbindung an Prelude-Manager (ActiveRecord)
    \item MSF-Plugin, um Payload von Exploits auf Paketebene zu loggen
    \item Verwendung von TCPDump + libpcap, um Payload von NMap zu loggen/verarbeiten
    \item Ergebnisdarstellung in GUI
    \vspace*{1em}
    \pause
    \item offene Tickets \textit{(nackte Zahlen)}:
    \begin{itemize}
      \item Fehler: 4/7
      \begin{itemize}
        \item seit letztem Freitag in Bearbeitung\dots
      \end{itemize}
      \item Feature: 5/6
      \begin{itemize}
        \item davon 2 Reconnaisance
      \end{itemize}
      \item ToDo: 8/24
      \begin{itemize}
        \item 3 Plumbing
        \item 1 Meta-ToDo (gruppenübergreifend mit Autopwn)
      \end{itemize}
    \end{itemize}
  \end{itemize}
\end{frame}

\section{Ausblick}
\secframe{Ausblick}{
  \begin{itemize}
    \item MSF-Worker stärker modularisieren
    \vspace*{1ex}
    \item Integration der CVE-DB
    \vspace*{1ex}
    \item Hosts und Pivots visualisieren
    \vspace*{1ex}
    \item Anpassungen an die FIDeS-Testlandschaft
    \vspace*{1ex}
    \item gewonnene Information verwenden, um z.B. »leiser« anzugreifen
  \end{itemize}
}

\section{Retrospektive}
\secframe{Retrospektive}{
  \begin{itemize}
    \item Einstieg in war Git nicht \textit{ganz} leicht
    \begin{itemize}
      \item in Branch gesteckte Arbeit nach Merge verschwunden\dots
      \item mehr Git-Repos erstellen und diese in der Mailingliste ankündigen\\
      (s.a. \url{https://gist.github.com/2ebb88f4e3442832ac50})
    \end{itemize}
    \vspace*{1ex}
    \item Redmine wieder intensiver in Benutzung
    \vspace*{1ex}
    \item Eindruck: kleinere Teams dynamischer und schaffen mehr \textit{!?}
    \begin{itemize}
      \item Absprachen einfacher + Terminfindung leichter
      \item niemand kann sich \textit{verstecken}
    \end{itemize}
  \end{itemize}
}

\end{document}

